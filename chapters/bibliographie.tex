\chapter*{Bibliographie}
\addcontentsline{toc}{chapter}{Bibliographie}

% Configuration for URL line breaking
\def\UrlBreaks{\do\/\do-\do.\do=\do_}
\urlstyle{same}
\Urlmuskip=0mu plus 1mu

\begin{itemize}[leftmargin=1cm, itemsep=0.3em, parsep=0.2em]
    \item[1.] Jeff Sutherland and Ken Schwaber, "Scrum: A Comprehensive Guide to Agile Project Management", 3rd Edition, 2022.
    \item[2.] Guide de démarrage Scrum | L'Agiliste.
    \item[3.] \url{https://www.redhat.com/en/topics/integration/whats-the-difference-between-soaprest}
    \item[4.] \url{https://www.researchgate.net/figure/Client-server-architecture-and-technologies_fig1_353325675}
    \item[5.] \url{https://nextjs.org/docs} - Next.js Documentation, framework React pour applications web modernes.
    \item[6.] \url{https://fastapi.tiangolo.com/} - FastAPI, un framework web Python pour la construction d'API rapides et performantes.
    \item[7.] \url{https://www.selenium.dev/documentation/} - Selenium WebDriver Documentation pour l'automatisation des navigateurs web.
    \item[8.] \url{https://www.keycloak.org/documentation} - Keycloak, solution d'authentification et d'autorisation open source.
    \item[9.] \url{https://spring.io/projects/spring-cloud-gateway} - Spring Cloud Gateway pour la gestion des API Gateway.
    \item[10.] \url{https://huggingface.co/cardiffnlp/twitter-xlm-roberta-\\base-sentiment} - Modèle cardiffnlp/twitter-xlm-roberta-base-sentiment pour l'analyse de sentiments multilingue.
    \item[11.] \url{https://huggingface.co/} - Hugging Face, plateforme pour les modèles de langage et les outils NLP.
    \item[12.] \url{https://www.docker.com/} - Docker, plateforme de conteneurisation pour le déploiement d'applications.
    \item[13.] \url{https://kubernetes.io/} - Kubernetes, système d'orchestration de conteneurs open source.
    \item[14.] \url{https://prometheus.io/} - Prometheus, système de monitoring et d'alerting open source.
    \item[15.] \url{https://grafana.com/} - Grafana, plateforme de visualisation et de monitoring.
    \item[16.] \url{https://www.postgresql.org/} - PostgreSQL, système de gestion de base de données relationnelle.
    \item[17.] \url{https://redis.io/} - Redis, base de données en mémoire pour le cache et les sessions.
    \item[18.] \url{https://react.dev/} - React.js, bibliothèque JavaScript pour la construction d'interfaces utilisateur.
    \item[19.] \url{https://tailwindcss.com/} - Tailwind CSS, framework CSS utility-first pour le design moderne.
    \item[20.] \url{https://www.python.org/} - Python, langage de programmation pour le développement backend et l'IA.
    \item[21.] \url{https://pydantic.dev/} - Pydantic, validation de données pour Python avec FastAPI.
    \item[22.] \url{https://pandas.pydata.org/} - Pandas, bibliothèque Python pour l'analyse et la manipulation de données.
    \item[23.] \url{https://numpy.org/} - NumPy, bibliothèque Python pour le calcul scientifique.
    \item[24.] \url{https://scikit-learn.org/} - Scikit-learn, bibliothèque d'apprentissage automatique pour Python.
    \item[25.] \url{https://pytorch.org/} - PyTorch, framework de deep learning pour Python.
    \item[26.] \url{https://www.nltk.org/} - NLTK, Natural Language Toolkit pour le traitement du langage naturel.
    \item[27.] \url{https://spacy.io/} - spaCy, bibliothèque avancée de traitement du langage naturel.
    \item[28.] \url{https://transformers.huggingface.co/} - Transformers, bibliothèque d'état de l'art pour les modèles NLP.
    \item[29.] \url{https://www.elastic.co/elasticsearch/} - Elasticsearch, moteur de recherche et d'analyse distribué.
    \item[30.] \url{https://www.jenkins.io/} - Jenkins, serveur d'automatisation pour CI/CD.
    \item[31.] \url{https://git-scm.com/} - Git, système de contrôle de version distribué.
    \item[32.] \url{https://github.com/} - GitHub, plateforme de développement collaboratif basée sur Git.
    \item[33.] \url{https://axios-http.com/} - Axios, client HTTP pour JavaScript et Node.js.
    \item[34.] \url{https://socket.io/} - Socket.IO, bibliothèque pour la communication temps réel.
    \item[35.] \url{https://jwt.io/} - JSON Web Tokens, standard pour l'authentification sécurisée.
    \item[36.] \url{https://oauth.net/2/} - OAuth 2.0, protocole d'autorisation pour les API.
    \item[37.] \url{https://openapi-generator.tech/} - OpenAPI Generator pour la génération de code API.
    \item[38.] \url{https://swagger.io/} - Swagger/OpenAPI, spécification pour documenter les API REST.
    \item[39.] \url{https://www.postman.com/} - Postman, plateforme pour le développement et les tests d'API.
    \item[40.] \url{https://insomnia.rest/} - Insomnia, client REST pour tester les API.
    \item[41.] Barbier, Guillaume, et al. "Analyse de sentiments et opinion mining sur les réseaux sociaux." Hermès Science Publications, 2018.
    \item[42.] Liu, Bing. "Sentiment Analysis: Mining Opinions, Sentiments, and Emotions." Cambridge University Press, 2020.
    \item[43.] Jurafsky, Dan, and James H. Martin. "Speech and Language Processing." 3rd Edition, 2023.
    \item[44.] Eisenstein, Jacob. "Introduction to Natural Language Processing." MIT Press, 2019.
    \item[45.] Goodfellow, Ian, et al. "Deep Learning." MIT Press, 2016.
    \item[46.] Chollet, François. "Deep Learning with Python." 2nd Edition, Manning Publications, 2021.
    \item[47.] Vaswani, Ashish, et al. "Attention is All You Need." NIPS 2017. (Transformer architecture)
    \item[48.] Devlin, Jacob, et al. "BERT: Pre-training of Deep Bidirectional Transformers for Language Understanding." NAACL 2019.
    \item[49.] Liu, Yinhan, et al. "RoBERTa: A Robustly Optimized BERT Pretraining Approach." arXiv preprint arXiv:1907.11692, 2019.
    \item[50.] Conneau, Alexis, et al. "Unsupervised Cross-lingual Representation Learning at Scale." ACL 2020. (XLM-RoBERTa)
    \item[51.] \url{https://hespress.com/} - Hespress, site d'actualités marocain source des données analysées.
    \item[52.] \url{https://www.code212.com/} - Code 212, centre de formation en technologies numériques au Maroc.
    \item[53.] Fowler, Martin. "Microservices Architecture." O'Reilly Media, 2021.
    \item[54.] Newman, Sam. "Building Microservices: Designing Fine-Grained Systems." 2nd Edition, O'Reilly Media, 2021.
    \item[55.] Richardson, Chris. "Microservices Patterns: With examples in Java." Manning Publications, 2018.
    \item[56.] \url{https://12factor.net/} - The Twelve-Factor App, méthodologie pour les applications SaaS.
    \item[57.] Burns, Brendan, and Joe Beda. "Kubernetes: Up and Running." 3rd Edition, O'Reilly Media, 2022.
    \item[58.] Hightower, Kelsey, et al. "Kubernetes: Up and Running." O'Reilly Media, 2019.
    \item[59.] Nickoloff, Jeff, and Stephen Kuenzli. "Docker in Action." 2nd Edition, Manning Publications, 2019.
    \item[60.] Miell, Ian, and Aidan Hobson Sayers. "Docker in Practice." 2nd Edition, Manning Publications, 2019.
    \item[61.] \url{https://www.atlassian.com/agile/scrum} - Guide Atlassian sur la méthodologie Scrum.
    \item[62.] \url{https://agilemanifesto.org/} - Manifeste Agile pour le développement logiciel.
    \item[63.] Cohn, Mike. "User Stories Applied: For Agile Software Development." Addison-Wesley Professional, 2004.
    \item[64.] Beck, Kent, et al. "Extreme Programming Explained: Embrace Change." 2nd Edition, Addison-Wesley Professional, 2004.
    \item[65.] \url{https://vercel.com/docs} - Vercel, plateforme de déploiement pour applications Next.js.
    \item[66.] \url{https://docs.python.org/3/} - Documentation officielle Python 3.
    \item[67.] \url{https://developer.mozilla.org/en-US/docs/Web/JavaScript} - Documentation JavaScript MDN.
    \item[68.] \url{https://www.typescriptlang.org/} - TypeScript, sur-ensemble typé de JavaScript.
    \item[69.] \url{https://helm.sh/} - Helm, gestionnaire de paquets pour Kubernetes.
    \item[70.] \url{https://istio.io/} - Istio, service mesh pour microservices.
\end{itemize}