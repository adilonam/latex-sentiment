\chaptertoc{Introduction generale}

\lettrine[nindent=0em, slope=-.5em] {\color{Eblue}A}{u} cours des deux dernières decennies, les avancees sans precedent dans le domaine des technologies de l'information ont engendre une veritable revolution, bouleversant profondement notre mode de vie et notre façon de travailler. Cependant, ce dynamisme technologique rapide n'est pas sans consequences. Il pose des defis sociaux, economiques et ethiques qui necessitent une reflexion approfondie et des adaptations constantes. Neanmoins, il est indeniable que les progrès technologiques ont ameliore notre qualite de vie de manière inimaginable, facilitant les tâches quotidiennes, stimulant la creativite, et ouvrant des perspectives sans limites.
\\

La gestion de la formation et de l'assistance technique represente toujours un defi considerable pour les centres de formation, surtout sans l'outil informatique adequat. La tâche devient alors plus complexe et coûteuse en raison des pertes de ressources et de temps liees à une gestion inefficace. L'informatique a prouve son importance en offrant des solutions efficaces pour la gestion des processus pedagogiques, de l'assistance technique, et des services divers, apportant un gain de temps et une simplification des tâches.
\\
L'introduction des chatbots intelligents, alimentes par l'intelligence artificielle (IA), apporte des benefices multiples dans divers domaines, y compris l'education et la formation. Ces assistants virtuels automatises peuvent interagir avec les utilisateurs de manière conversationnelle, offrant ainsi une assistance instantanee et personnalisee. Les applications des chatbots en general incluent :
\begin{itemize}
    \item \textbf{Assistance Instantanee} : Les etudiants beneficient d'une aide immediate pour leurs questions frequentes, ce qui facilite un apprentissage autonome et renforce le processus educatif.
    \item \textbf{Amelioration de l'Employabilite} : En fournissant des outils pratiques et interactifs pour la formation, Code212 prepare mieux ses etudiants à integrer le marche du travail numerique.
    \item \textbf{Optimisation des Ressources} : Le chatbot aide à une utilisation plus efficace des ressources pedagogiques disponibles, maximisant ainsi l'efficacite des etudes des apprenants.
    \item \textbf{Support Moral} : En offrant un soutien empathique et motivant, le chatbot contribue à maintenir la perseverance et l'engagement des etudiants.
\end{itemize}
\ \\

C'est dans ce contexte que notre projet de fin d'annee a vu le jour, au sein du centre Code212 de l'Universite Ibn Zohr. Le projet consiste à elaborer un assistant chatbot intelligent et à l'integrer dans une plateforme e-learning pour le centre Code212. Ces outils visent à ameliorer l'efficacite de la formation et de l'assistance technique, en permettant une interaction centralisee et accessible via Internet.
\\

Pour repondre au besoin, nous avons opte pour la methode Scrum pour notre projet. Cette methode favorise une flexibilite et une adaptation continues aux exigences changeantes du client. Nous avons utilise des outils et frameworks d'IA tels que Hugging Face, Transformers, LLaMA2 et LangChain pour renforcer les capacites de notre chatbot intelligent. En complement, nous avons utilise le framework Spring Boot, Next.js et FastAPI, completes par Tailwind CSS pour la conception de l'interface utilisateur. Cette combinaison de technologies nous a permis de developper des solutions performantes et efficaces.
