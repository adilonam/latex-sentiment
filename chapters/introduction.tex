\chaptertoc{Introduction generale}

\lettrine[nindent=0em, slope=-.5em] {\color{Eblue}A}{u} cours des deux dernières decennies, les avancees sans precedent dans le domaine des technologies de l'information ont engendre une veritable revolution, bouleversant profondement notre mode de vie et notre façon de travailler. Cette transformation s'est particulièrement manifestee dans l'explosion des reseaux sociaux qui sont devenus des plateformes incontournables d'expression et d'echange d'opinions. Neanmoins, il est indeniable que ces progrès technologiques ont ameliore notre capacite à communiquer et à partager nos sentiments, facilitant les interactions sociales et ouvrant des perspectives sans limites pour l'analyse des comportements humains.
\\

L'analyse des sentiments exprimes sur les reseaux sociaux represente aujourd'hui un defi considerable pour les entreprises, les institutions et les chercheurs. La masse colossale de donnees textuelles generees quotidiennement sur ces plateformes rend l'analyse manuelle impossible et coûteuse. L'intelligence artificielle a prouve son importance en offrant des solutions efficaces pour le traitement automatique du langage naturel, l'analyse des emotions et l'extraction d'insights meaningfuls à partir des commentaires des utilisateurs.
\\
L'analyse de sentiment des commentaires sur les reseaux sociaux, alimentee par l'intelligence artificielle (IA), apporte des benefices multiples dans divers domaines, y compris le marketing, la recherche sociale et la gestion de la reputation. Ces systemes automatises peuvent analyser et classifier les emotions exprimees dans les textes, offrant ainsi une comprehension approfondie de l'opinion publique. Les applications de l'analyse de sentiment incluent :
\begin{itemize}
    \item \textbf{Veille Strategique} : Les entreprises beneficient d'une analyse en temps reel de l'opinion des consommateurs sur leurs produits et services, ce qui facilite la prise de decisions strategiques.
    \item \textbf{Gestion de la Reputation} : En detectant rapidement les sentiments negatifs, les organisations peuvent reagir promptement pour proteger leur image de marque.
    \item \textbf{Recherche en Sciences Sociales} : L'analyse automatique des sentiments permet aux chercheurs d'etudier les tendances sociales et les reactions du public à grande echelle.
    \item \textbf{Amelioration des Services} : En analysant les retours des utilisateurs, les entreprises peuvent identifier les points d'amelioration et optimiser leurs offres.
\end{itemize}
\ \\

C'est dans ce contexte que notre projet de fin d'étude a vu le jour, visant à developper une application d'analyse de sentiment des commentaires du site d'actualites marocain Hespress avec l'intelligence artificielle. Ce projet consiste à elaborer un système intelligent capable de collecter automatiquement les commentaires des articles via web scraping, puis d'analyser et classifier les sentiments exprimes par les lecteurs. L'objectif est d'offrir aux medias et aux chercheurs un outil performant pour comprendre l'opinion publique marocaine et les tendances emotionnelles autour des actualites nationales.
\\

Pour repondre à ce besoin, nous avons opte pour une approche technologique moderne et efficace. Pour la collecte des donnees, nous avons utilise Selenium, un framework de web scraping puissant qui nous permet d'extraire automatiquement les commentaires des articles de Hespress en naviguant dans les pages web de manière automatisee. Pour l'analyse de sentiment, nous avons implemente un modèle de classification de texte pre-entraine de Hugging Face, une plateforme leader dans le domaine du traitement du langage naturel. Ce modèle nous permet d'analyser et de classifier les sentiments (positif, negatif, neutre) exprimes dans les commentaires collectes. Cette combinaison de technologies nous a permis de developper une solution performante et precise pour l'analyse de sentiment des actualites marocaines en temps reel.
