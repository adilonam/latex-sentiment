\chapter{Presentation du projet :}

\section{Problematique et presentation du projet}
Avec l'explosion des medias numeriques et des plateformes d'information en ligne, l'analyse des sentiments exprimés par les utilisateurs dans leurs commentaires est devenue cruciale pour comprendre l'opinion publique. Face à ce besoin croissant, Code 212 a identifie l'opportunite de developper une application d'analyse de sentiments specialisee dans le traitement des commentaires de la plateforme Hespress, permettant d'extraire et d'analyser automatiquement les opinions et emotions exprimees par les lecteurs.

\subsection{Contexte des Medias Numeriques}
Le paysage mediatique marocain a connu une transformation majeure avec l'emergence de plateformes d'information numeriques comme Hespress. Ces plateformes generent quotidiennement des milliers de commentaires d'utilisateurs, constituant une mine d'informations precieuse sur l'opinion publique et les sentiments de la population.

\subsection{Besoin d'Analyse Automatisee des Sentiments}
L'analyse manuelle des commentaires sur les articles d'actualite etant impossible à grande echelle, il devient essentiel de disposer d'outils automatises capables de classifier et d'analyser les sentiments exprimés. Cette analyse permet aux journalistes, chercheurs et decideurs de mieux comprendre les reactions du public face aux evenements d'actualite.

\section{Solution Proposee}
L'application d'analyse de sentiments represente une innovation technologique pour Code 212. Capable d'extraire automatiquement les commentaires de Hespress et de les analyser en temps reel, elle offre une comprehension approfondie des opinions publiques tout en optimisant le processus d'analyse de donnees textuelles.

\subsection{L'Application d'Analyse de Sentiments}
L'application, dotee d'intelligence artificielle avancee, est concue pour scraper automatiquement les commentaires de la plateforme Hespress, les traiter à travers un modèle de classification de sentiments pre-entraine, et presenter les resultats à travers une interface utilisateur moderne et intuitive.

\subsection{Architecture Technologique}
Notre solution s'appuie sur une architecture microservices moderne et robuste, integrant plusieurs technologies de pointe pour assurer performance, securite et scalabilite.
\subsubsection{Frontend : Next.js}
Next.js constitue le framework de choix pour le developpement de l'interface utilisateur de notre application.

\paragraph{Avantages de Next.js}
\begin{itemize}
    \item \textbf{Rendu hybride}: Combine le rendu côte serveur (SSR) et la generation statique (SSG) pour des performances optimales.
    \item \textbf{Optimisation automatique}: Optimisation automatique des images, du code et des performances pour une experience utilisateur fluide.
    \item \textbf{Routing simplifie}: Système de routage base sur le système de fichiers, facilitant la navigation et l'organisation du code.
\end{itemize}

\subsubsection{Backend : FastAPI}
FastAPI sert de framework principal pour le developpement de l'API backend de notre application.

\paragraph{Avantages de FastAPI}
\begin{itemize}
    \item \textbf{Performance elevee}: Performances comparables à NodeJS et Go, grâce à la programmation asynchrone.
    \item \textbf{Documentation automatique}: Generation automatique de documentation interactive avec Swagger UI.
    \item \textbf{Validation automatique}: Validation automatique des donnees avec des messages d'erreur detailles.
\end{itemize}

\subsubsection{Authentification : Keycloak}
Keycloak assure la gestion complete de l'authentification et de l'autorisation de notre application.

\paragraph{Avantages de Keycloak}
\begin{itemize}
    \item \textbf{Securite enterprise}: Solution d'authentification robuste avec support des standards OAuth 2.0 et OpenID Connect.
    \item \textbf{Gestion centralisee}: Gestion centralisee des utilisateurs, roles et permissions.
    \item \textbf{Scalabilite}: Solution adaptee aux applications à grande echelle avec support du clustering.
\end{itemize}

\subsubsection{Web Scraping : Selenium}
Selenium permet l'extraction automatisee des commentaires depuis la plateforme Hespress.

\paragraph{Avantages de Selenium}
\begin{itemize}
    \item \textbf{Automation complete}: Automatisation complete des interactions avec les pages web dynamiques.
    \item \textbf{Support multi-navigateurs}: Compatibilite avec tous les navigateurs modernes.
    \item \textbf{Gestion du JavaScript}: Capacite à traiter les contenus generes dynamiquement par JavaScript.
\end{itemize}

\subsubsection{Classification de Sentiments : CardiffNLP Twitter-XLM-RoBERTa}
Le modèle CardiffNLP/twitter-xlm-roberta-base-sentiment constitue le cœur de notre système d'analyse de sentiments.

\paragraph{Avantages du Modèle}
\begin{itemize}
    \item \textbf{Multilinguisme}: Modèle pre-entraine sur des donnees multilingues, ideal pour les textes en arabe, francais et anglais.
    \item \textbf{Robustesse}: Architecture RoBERTa offrant une comprehension contextuelle avancee des textes.
    \item \textbf{Specialisation sociale}: Entraine specifiquement sur des donnees de reseaux sociaux, adapte aux commentaires en ligne.
\end{itemize}

\subsubsection{API Gateway : Spring Cloud Gateway}
Spring Cloud Gateway orchestre les communications entre les differents microservices de notre architecture.

\paragraph{Avantages de Spring Gateway}
\begin{itemize}
    \item \textbf{Routage intelligent}: Routage dynamique des requêtes vers les microservices appropries.
    \item \textbf{Resilience}: Mecanismes de circuit breaker et de retry pour une haute disponibilite.
    \item \textbf{Observabilite}: Integration native avec les outils de monitoring et de tracing distribue.
\end{itemize}


\section{Tableau des fonctionnalitees}
Nous presentons ici un tableau decrivant les fonctionnalites offertes par l'application d'analyse de sentiments et leur impact sur l'experience utilisateur.


\begin{table}[h!]
\centering
\begin{tabular}{|p{6cm}|p{10cm}|}
\hline
\textbf{Fonctionnalite} & \textbf{Description et Impact} \\ \hline
Extraction automatique de commentaires & L'application utilise Selenium pour extraire automatiquement les commentaires des articles Hespress, permettant une collecte de donnees en temps reel sans intervention manuelle. \\ \hline
Analyse de sentiments en temps reel & Grace au modèle CardiffNLP, l'application classifie instantanement les commentaires en sentiments positifs, negatifs ou neutres, offrant une comprehension immediate de l'opinion publique. \\ \hline
Interface utilisateur intuitive & L'interface Next.js propose une navigation fluide et des visualisations claires des resultats d'analyse, facilitant l'interpretation des donnees pour les utilisateurs finaux. \\ \hline
Authentification securisee & L'integration Keycloak assure une gestion robuste des utilisateurs et des permissions, garantissant la securite et la confidentialite des donnees analysees. \\ \hline
Architecture microservices & La separation en microservices avec Spring Gateway permet une scalabilite optimale et une maintenance facilitee du système. \\ \hline
\end{tabular}
\caption{Recapitulatif des fonctionnalites de l'application d'analyse de sentiments Hespress.}
\label{tab:sentimentfeatures}
\end{table}

\section{Conduite du Projet}
La conduite efficace du projet de developpement de l'application d'analyse de sentiments est essentielle pour repondre aux exigences specifiques de Code 212. Cette section explore en detail les processus de conception, de planification et de developpement qui sous-tendent la creation de cette solution innovante.

\subsection{Conception}
La phase de conception est l'etape fondatrice du projet, où l'equipe fixe les objectifs de l'application d'analyse de sentiments, identifie les fonctionnalites cles et elabore l'architecture generale du système. Cette phase necessite une comprehension precise des besoins en analyse de donnees textuelles et des attentes des utilisateurs finaux. À ce stade, des ateliers de brainstorming, des etudes de marche sur les plateformes d'actualite et des seances de feedback avec les parties prenantes sont utilises pour rassembler les exigences et pour esquisser les premiers prototypes de l'interface utilisateur. Les scenarios d'utilisation pour l'extraction et l'analyse des commentaires sont meticuleusement elabores pour garantir que l'application est à la fois performante et techniquement viable.

\subsection{Planification des Tâches}
Une planification detaillee va de pair avec une conception reussie; elle s'attaque à la complexite du processus de developpement en ventilant le projet en petites tâches gerables. L'equipe du projet etablit un calendrier de realisation, determinant les dependances entre les differents composants (scraping, analyse IA, frontend, backend) et en allouant les ressources necessaires. La methodologie Agile est adoptee, permettant une flexibilite et une reactivite accrues face aux changements de scope ou aux evolutions des modèles d'IA. La creation de sprints focuses sur chaque technologie (Selenium, FastAPI, Next.js, etc.), la priorisation des backlogs et les revues regulières de sprint assurent que le projet reste sur la bonne voie et s'adapte aux exigences changeantes tout au long de son cycle de vie.

\subsection{Developpement et Integrations}
Durant la phase de developpement, l'equipe met en œuvre la conception de l'application en integrant les differentes composantes technologiques. Les developpeurs intègrent le modèle de classification CardiffNLP, implementent les mecanismes de scraping avec Selenium, developpent l'API avec FastAPI et creent l'interface utilisateur avec Next.js. L'integration de Keycloak pour l'authentification et de Spring Gateway pour l'orchestration des microservices est rigoureusement testee pour garantir son bon fonctionnement. Une serie exhaustive de tests—tests unitaires pour chaque composant, tests d'integration entre les microservices et tests de charge pour valider les performances du scraping et de l'analyse—est realisee pour s'assurer que le système est robuste, evolutif et prêt pour le deploiement.
Chacune de ces etapes est critique pour le succès de l'application d'analyse de sentiments qui est censee non seulement extraire et analyser efficacement les commentaires Hespress mais aussi evoluer avec les besoins du centre de formation Code 212 afin d'ameliorer continuellement les capacites d'analyse offertes.
