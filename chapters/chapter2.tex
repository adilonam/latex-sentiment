\chapter{Presentation du projet :}

\section{Problematique et presentation du projet}
Avec la montee en puissance de la digitalisation, les etablissements educatifs doivent constamment s'adapter pour offrir à leurs etudiants des outils adaptes à leur reussite. Face à cet imperatif, Code 212 a identifie le besoin d'assurer une interaction constante et personnalisee afin de repondre efficacement aux questions frequentes des etudiants et de les assister dans leur parcours educatif.

\subsection{Contexte d'Apprentissage Numerique}
Le contexte actuel de l'education est marque par un passage rapide au numerique. Cela requiert des moyens d'apprentissage qui soient non seulement flexibles mais aussi immediatement reactifs aux besoins des elèves.

\subsection{Besoin d'une Interaction Continue}
À l'ère du numerique, où l'immediatete est devenue la norme, les etudiants s'attendent à obtenir rapidement des reponses à leurs questions. Un outil tel qu'un chatbot IA s'avère indispensable pour combler cet ecart entre les besoins des etudiants et les ressources disponibles.

\section{Solution Proposee}
Le chatbot IA represente une innovation centrale pour Code 212. Capable de dialoguer avec les etudiants en temps reel, il offre une assistance instantanee tout en optimisant les ressources pedagogiques du centre.

\subsection{Le Chatbot IA}
Le chatbot, dote d'intelligence artificielle, est programme pour comprendre et traiter les demandes des etudiants, fournissant des reponses claires et precises, et est même capable d'apprendre de ses interactions pour ameliorer continuellement son assistance.

\subsection{Methodes de Developpement du Chatbot}
Lors du developpement du chatbot, nous avions deux options principales : le fine-tuning et le Retrieval-Augmented Generation (RAG). Après une analyse approfondie des avantages et des inconvenients de chaque methode, nous avons opte pour la methode RAG.

\subsubsection{Le Fine-Tuning}
Le fine-tuning consiste à adapter un modèle de langage pre-entraîne à des tâches specifiques en le re-entraînant sur un ensemble de donnees supplementaires pertinentes.

\paragraph{Avantages du Fine-Tuning}
\begin{itemize}
    \item \textbf{Adaptation specifique}: Permet d'adapter le modèle aux besoins precis en utilisant des donnees specifiques à notre domaine.
    \item \textbf{Amelioration des performances}: Peut ameliorer la precision du modèle pour des tâches specifiques en integrant des nuances et des contextes particuliers.
\end{itemize}

\paragraph{Inconvenients du Fine-Tuning}
\begin{itemize}
    \item \textbf{Coût en ressources}: Le processus de fine-tuning peut être coûteux en termes de calcul et de temps, necessitant des ressources materielles importantes.
    \item \textbf{Dependance aux donnees}: La qualite des resultats depend fortement de la qualite et de la quantite des donnees disponibles pour le re-entraînement.
    \item \textbf{Maintenance complexe}: Necessite une mise à jour continue et une maintenance pour integrer les nouvelles informations et rester pertinent.
\end{itemize}

\subsubsection{Le Retrieval-Augmented Generation (RAG)}
RAG combine un modèle de recuperation d'informations et un modèle de generation de texte. Il fonctionne en recuperant des passages pertinents d'une base de donnees et en generant des reponses en s'appuyant sur ces passages.

\paragraph{Avantages du RAG}
\begin{itemize}
    \item \textbf{Accès à une large base de connaissances}: RAG permet d'acceder à une base de donnees etendue pour recuperer les informations les plus pertinentes, offrant ainsi des reponses plus precises et à jour.
    \item \textbf{Reduction des coûts de calcul}: etant donne que le modèle genère des reponses basees sur des informations recuperees, il necessite moins de re-entraînement intensif, ce qui reduit les coûts en calcul et en temps.
    \item \textbf{Scalabilite et flexibilite}: Facile à mettre à jour avec de nouvelles informations sans necessiter un re-entraînement complet du modèle.
\end{itemize}

\paragraph{Inconvenients du RAG}
\begin{itemize}
    \item \textbf{Complexite de mise en œuvre}: Integrer et optimiser un système RAG peut être plus complexe initialement par rapport au fine-tuning simple d'un modèle.
    \item \textbf{Dependance à la base de donnees}: La qualite des reponses depend de la qualite et de la couverture de la base de donnees utilisee pour la recuperation des informations.
\end{itemize}

\subsubsection{Pourquoi Nous Avons Choisi le RAG}
Nous avons choisi la methode RAG en raison de ses nombreux avantages strategiques. La capacite d'acceder à une vaste base de connaissances et de fournir des reponses precises et actualisees etait cruciale pour nous. De plus, la reduction des coûts de calcul et la flexibilite de mise à jour ont ete des facteurs determinants, permettant une maintenance plus aisee et des mises à jour regulières sans necessiter un re-entraînement complet. Bien que la mise en œuvre initiale ait ete plus complexe, les benefices à long terme en termes de scalabilite et d'efficacite ont justifie notre choix pour la methode RAG.


\section{Tableau des fonctionnalitees}
Nous presentons ici un tableau decrivant les solutions offertes par le chatbot et leur impact sur l'experience des etudiants.


\begin{table}[h!]
\centering
\begin{tabular}{|p{6cm}|p{10cm}|}
\hline
\textbf{Fonctionnalite} & \textbf{Description et Impact sur les etudiants} \\ \hline
Reponses aux questions frequentes & Le chatbot fournit des reponses immediates aux interrogations des etudiants, facilitant l'apprentissage autonome et une meilleure comprehension des cours. \\ \hline
Assistance pour les travaux pratiques & Il guide les etudiants dans la realisation de leurs travaux pratiques en offrant des conseils et des ressources utiles, ce qui contribue à approfondir leur comprehension et à ameliorer leur competence pratique. \\ \hline
Gestion des ressources pedagogiques & Le chatbot aide à la navigation et à l'exploitation efficace des ressources pedagogiques mises à disposition, maximisant ainsi le temps d'etude et les resultats d'apprentissage. \\ \hline
Support psychologique & À travers des interactions engageantes et empathiques, le chatbot peut proposer un soutien moral et motiver les etudiants, renforçant leur perseverance et leur engagement scolaire. \\ \hline
\end{tabular}
\caption{Recapitulatif des fonctionnalites du chatbot IA et leur impact sur les etudiants de Code 212.}
\label{tab:chatbotfeatures}
\end{table}

\section{Conduite du Projet}
La conduite efficace du projet de developpement du chatbot IA est essentielle pour repondre aux exigences specifiques de Code 212. Cette section explore en detail les processus de conception, de planification et de developpement qui sous-tendent la creation du chatbot.

\subsection{Conception}
La phase de conception est l'etape fondatrice du projet, où l'equipe fixe les objectifs du chatbot, identifie les fonctionnalites cles et elabore la structure generale de l'intelligence artificielle. Cette phase necessite une comprehension precise des besoins des etudiants qui seront les utilisateurs finaux. À ce stade, des ateliers de brainstorming, des enquêtes auprès des utilisateurs et des seances de feedback avec les principales parties prenantes sont utilises pour rassembler les exigences et pour esquisser les premiers prototypes de l'interface utilisateur. Les scenarios d'utilisation et les parcours utilisateurs sont meticuleusement elabores pour garantir que le chatbot est à la fois convivial et techniquement viable.

\subsection{Planification des Tâches}
Une planification detaillee va de pair avec une conception reussie; elle s'attaque à la complexite du processus de developpement en ventilant le projet en petites tâches gerables. L'equipe du projet etablit un calendrier de realisation, determinant les dependances entre les tâches et en allouant les ressources necessaires. La methodologie Agile est souvent adoptee, permettant une flexibilite et une reactivite accrues face aux changements de scope ou aux retours des utilisateurs. La creation de sprints, la priorisation des backlogs et les revues regulières de sprint assurent que le projet reste sur la bonne voie et adapte aux exigences changeantes tout au long de son cycle de vie.

\subsection{Developpement et Integrations}
Durant la phase de developpement, l'equipe met en œuvre la conception du chatbot en codant les differentes composantes logicielles. Les developpeurs intègrent l'intelligence artificielle, travaillent sur le traitement du langage naturel et l'apprentissage automatique pour offrir une experience utilisateur la plus authentique possible. La construction de l'infrastructure technique, comprenant des serveurs, des bases de donnees et l'integration avec les API existantes, est rigoureusement testee pour garantir son bon fonctionnement. Une serie exhaustive de tests—tests unitaires, tests d'integration et tests de charge—est realisee pour s'assurer que le système est robuste, evolutif et prêt pour le deploiement.
Chacune de ces etapes est critique pour le succès du chatbot IA qui est cense non seulement interagir efficacement avec les utilisateurs mais aussi evoluer avec le centre educatif Code 212 afin d'ameliorer continuellement le support offert aux etudiants.
