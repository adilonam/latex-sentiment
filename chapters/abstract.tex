\chaptertoc{Abstract}

\lettrine[nindent=0em, slope=.5em] {\color{Eblue}T}{his} document presents a synthesis of our end-of-year project conducted at \textbf{Code212}. This innovative project focuses on developing a sentiment analysis application capable of analyzing and understanding emotions and opinions expressed in textual comments.

\ \\

\lettrine[nindent=0em, slope=.5em] {\color{Eblue}T}{he} project involves designing and implementing a sophisticated web application that utilizes \textbf{artificial intelligence}, including an AI process pipeline with LLMs and text classification, to evaluate the emotional tone of comments. The application allows users to submit textual comments and provides detailed analysis of their sentiment, categorizing texts as positive, negative, or neutral. The system is also capable of identifying emotional nuances and main themes present in the comments.

\ \\

\lettrine[nindent=0em, slope=.5em] {\color{Eblue}T}{he} application incorporates advanced \textbf{Natural Language Processing} technologies and machine learning to ensure precise and nuanced analysis. It provides an intuitive user interface for visualizing analysis results through graphs and detailed reports, facilitating the understanding of emotional trends in comments.

\ \\

\lettrine[nindent=0em, slope=.5em] {\color{Eblue}T}{o} manage this project effectively, we adopted the \textbf{SCRUM} methodology and divided the work into sprints. An initial sprint was dedicated to researching and selecting the most appropriate AI algorithms, followed by the development of the \textbf{sentiment analysis model} and the creation of the user interface. Each sprint was meticulously planned to ensure coherent project progression.

\ \\

\lettrine[nindent=0em, slope=.5em] {\color{Eblue}I}{n} summary, our sentiment analysis application developed for Code212 represents an innovative solution for automatically understanding and analyzing emotions expressed in textual comments. It combines advanced AI technologies with an intuitive user interface, offering a powerful tool for opinion and sentiment analysis in comments.
