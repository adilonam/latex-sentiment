\chaptertoc{Resume}

\lettrine[nindent=0em, slope=-.5em] {\color{Eblue}L}{e} present document est une synthèse de notre projet de fin d'annee realise au sein de l'entreprise \textbf{Code212}. Durant ce stage, nous avons eu la chance de travailler sur un projet informatique visant à integrer un chatbot intelligent dans la plateforme e-learning de \textbf{Code212}.

\ \\

\lettrine[nindent=0em, slope=-.5em] {\color{Eblue}L}{e} projet consiste à creer et à integrer un chatbot AI dans une application web dediee à la formation en ligne et la gestion des cours et certificats pour les etudiants. Les utilisateurs de la plateforme incluent des etudiants, des gestionnaires et des administrateurs. Les etudiants pourront interagir avec le chatbot pour obtenir des informations sur les cours, s'inscrire à des evenements et examens de certification, et recevoir de l'aide personnalisee. Les gestionnaires auront la possibilite de creer, modifier et supprimer des cours, evenements et certificats. Les administrateurs seront charges de gerer les utilisateurs et d'assurer le bon fonctionnement de la plateforme.

\ \\

\lettrine[nindent=0em, slope=-.5em] {\color{Eblue}L}{e} chatbot AI est conçu pour ameliorer l'efficacite et l'accessibilite des services de formation en ligne, offrant une assistance immediate et personnalisee aux utilisateurs. Grâce à cette integration, les etudiants beneficieront d'une interface interactive et intuitive, facilitant leur parcours d'apprentissage.

\ \\

\lettrine[nindent=0em, slope=-.5em] {\color{Eblue}A}{fin} de gerer au mieux ce projet, nous avons opte pour la methode \textbf{SCRUM} et divise les differents modules en sprints. Un sprint de preparation a ete initie pour effectuer une analyse fonctionnelle globale et construire un modèle de base pour les sprints suivants. Après avoir termine le sprint de preparation, nous avons execute les autres sprints, chacun etant dedie à la conception detaillee d'un module de l'application, suivi par l'elaboration de ses maquettes et enfin sa realisation.

\ \\

\lettrine[nindent=0em, slope=-.5em] {\color{Eblue}E}{n} resume, le projet vise à integrer un chatbot AI dans une plateforme e-learning, offrant une experience d'apprentissage amelioree pour les etudiants et une gestion efficace des cours et evenements pour les gestionnaires et administrateurs. Cette integration assure une interaction fluide et securisee, tout en augmentant l'accessibilite et l'efficacite des services educatifs en ligne de \textbf{Code212}.
