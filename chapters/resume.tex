\chaptertoc{Resume}

\lettrine[nindent=0em, slope=.5em] {\color{Eblue}L}{e} present document est une synthèse de notre projet de fin d'étude realise au sein du centre de formation \textbf{Code212}. Ce projet innovant de \textbf{classification de texte} vise à developper une application d'analyse de sentiment capable d'analyser et de comprendre les emotions et opinions exprimees dans les commentaires du site \textbf{Hespress}.

\ \\

\lettrine[nindent=0em, slope=.5em] {\color{Eblue}L}{e} projet consiste à concevoir et implementer une application web sophistiquee de \textbf{classification de texte} pour l'analyse de sentiment qui utilise une architecture moderne avec \textbf{Next.js} pour le frontend, \textbf{FastAPI} pour le backend, \textbf{Keycloak} pour l'authentification, \textbf{Selenium} pour le web scraping des commentaires Hespress, le modèle \textbf{cardiffnlp/twitter-xlm-roberta-base-sentiment} pour la classification de texte, et \textbf{Spring Cloud Gateway} comme passerelle API. L'application permet d'extraire automatiquement les commentaires d'Hespress et fournit une analyse detaillee de leur sentiment, categorisant les textes comme positifs, negatifs ou neutres.

\ \\

\lettrine[nindent=0em, slope=.5em] {\color{Eblue}L}{'application} integre des technologies avancees de \textbf{traitement du langage naturel} (NLP) et d'apprentissage automatique avec le modèle pre-entraine cardiffnlp pour garantir une analyse precise et nuancee des commentaires Hespress. Elle offre une interface utilisateur moderne developpe avec Next.js permettant de visualiser les resultats d'analyse sous forme de graphiques et de rapports detailles, facilitant ainsi la comprehension des tendances emotionnelles dans les commentaires extraits.

\ \\

\lettrine[nindent=0em, slope=.5em] {\color{Eblue}A}{fin} de gerer au mieux ce projet, nous avons opte pour la methode \textbf{SCRUM} et divise les differents modules en sprints. Le Sprint 1 a ete consacre à la configuration de l'environnement et l'integration du modèle d'analyse. Le Sprint 2 s'est concentre sur le developpement du web scraping avec Selenium et le preprocessing des donnees. Le Sprint 3 a porte sur le developpement du tableau de bord avec Next.js et l'interface d'authentification avec Keycloak. Enfin, le Sprint 4 a finalise la generation de rapports, l'optimisation et le deploiement avec Spring Cloud Gateway.

\ \\

\lettrine[nindent=0em, slope=.5em] {\color{Eblue}E}{n} resume, notre application d'analyse de sentiment developee pour le centre de formation Code212 represente une solution innovante de \textbf{classification de texte} pour comprendre et analyser automatiquement les emotions exprimees dans les commentaires du site Hespress. Elle combine des technologies modernes comme Next.js, FastAPI, Keycloak, Selenium et le modèle cardiffnlp avec une architecture microservices utilisant Spring Cloud Gateway, offrant un outil puissant pour l'analyse de l'opinion et du sentiment dans les commentaires d'actualites.
