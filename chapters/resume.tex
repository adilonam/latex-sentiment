\chaptertoc{Resume}

\lettrine[nindent=0em, slope=.5em] {\color{Eblue}L}{e} present document est une synthèse de notre projet de fin d'annee realise au sein de l'entreprise \textbf{Code212}. Ce projet innovant vise à developper une application d'analyse de sentiment capable d'analyser et de comprendre les emotions et opinions exprimees dans les commentaires textuels.

\ \\

\lettrine[nindent=0em, slope=.5em] {\color{Eblue}L}{e} projet consiste à concevoir et implementer une application web sophistiquee d'analyse de sentiment qui utilise des \textbf{intelligence artificielle} pour evaluer la tonalite emotionnelle des commentaires. L'application permet aux utilisateurs de soumettre des commentaires textuels et fournit une analyse detaillee de leur sentiment, categorisant les textes comme positifs, negatifs ou neutres. Le système est egalement capable d'identifier les nuances emotionnelles et les themes principaux presents dans les commentaires.

\ \\

\lettrine[nindent=0em, slope=.5em] {\color{Eblue}L}{'application} integre des technologies avancees de \textbf{traitement du langage naturel} (NLP) et d'apprentissage automatique pour garantir une analyse precise et nuancee. Elle offre une interface utilisateur intuitive permettant de visualiser les resultats d'analyse sous forme de graphiques et de rapports detailles, facilitant ainsi la comprehension des tendances emotionnelles dans les commentaires.

\ \\

\lettrine[nindent=0em, slope=.5em] {\color{Eblue}A}{fin} de gerer au mieux ce projet, nous avons opte pour la methode \textbf{SCRUM} et divise les differents modules en sprints. Un sprint initial a ete consacre à la recherche et à la selection des algorithmes d'IA les plus appropries, suivi par le developpement du \textbf{modèle d'analyse de sentiment} et la creation de l'interface utilisateur. Chaque sprint a ete minutieusement planifie pour assurer une progression coherente du projet.

\ \\

\lettrine[nindent=0em, slope=.5em] {\color{Eblue}E}{n} resume, notre application d'analyse de sentiment developee pour Code212 represente une solution innovante pour comprendre et analyser automatiquement les emotions exprimees dans les commentaires textuels. Elle combine des technologies d'IA avancees avec une interface utilisateur intuitive, offrant un outil puissant pour l'analyse de l'opinion et du sentiment dans les commentaires.
